% !TEX root = ../main.tex
\chapter{HASIL DAN PEMBAHASAN}
\label{chap:hasil}

\section{Gambaran Umum Eksperimen dan Lingkungan Pengujian}
Bab ini menyajikan hasil eksperimen dan pembahasan terhadap kinerja model deteksi objek, integrasi sistem pelacakan, serta performa sistem streaming end-to-end yang dikembangkan. Pengujian dilakukan untuk mengevaluasi apakah sistem yang diusulkan mampu memenuhi tujuan penelitian sebagaimana dirumuskan pada Bab I.

Seluruh eksperimen dilakukan pada lingkungan komputasi terpusat dengan spesifikasi perangkat keras dan perangkat lunak sebagaimana dirangkum pada Tabel~\ref{tab:env}.

\begin{table}[H]
\centering
\caption{Spesifikasi Lingkungan Pengujian}
\label{tab:env}
\begin{tabular}{ll}
\hline
\textbf{Komponen} & \textbf{Spesifikasi} \\ \hline
Sistem Operasi & Debian GNU/Linux 13 (Trixie) \\
CPU & Intel Core i5-5200U @ 2.20 GHz (2 Core / 4 Thread) \\
GPU & NVIDIA GeForce 940M (2 GB VRAM, CUDA Capability 5.0) \\
RAM & 12 GB DDR3L \\
Python & 3.13.5 \\
Framework Deteksi & PyTorch 2.10.0, Ultralytics YOLOv8 \\
Pelacakan & ByteTrack via Supervision 0.27.0 \\
Message Broker & Apache Kafka 3.9.1 \\
Streaming Engine & Apache Spark 3.5.8 (Structured Streaming) \\
Database & PostgreSQL 15 (Supabase) \\
Dashboard & Grafana OSS \\ \hline
\end{tabular}
\end{table}

Perlu dicatat bahwa GPU NVIDIA GeForce 940M memiliki \textit{CUDA Compute Capability} 5.0, yang tidak didukung oleh versi PyTorch 2.10.0 (mendukung minimal \textit{Compute Capability} 7.0). Oleh karena itu, seluruh proses inferensi pada eksperimen ini dilakukan secara \textit{fallback} pada CPU. Keterbatasan ini berdampak pada throughput sistem yang akan dibahas lebih lanjut pada Bagian~\ref{sec:throughput}.

Dataset yang digunakan dalam pengujian terdiri atas dataset lokal ITERA Helmet Dataset serta dataset eksternal berbasis CCTV publik Kota Palembang, sebagaimana dirangkum pada Tabel~\ref{tab:dataset}.

\begin{table}[H]
\centering
\caption{Ringkasan Dataset Pengujian}
\label{tab:dataset}
\begin{tabular}{llll}
\hline
\textbf{Dataset} & \textbf{Sumber} & \textbf{Jumlah Data} & \textbf{Peran} \\ \hline
ITERA Helmet Dataset & CCTV Gerbang Utama ITERA & -- anotasi & Train / Val / Test \\
CCTV Palembang & Open CCTV Kota Palembang & -- anotasi & Uji Generalisasi \\ \hline
\end{tabular}
\end{table}

Model YOLOv8 dilatih dengan tiga kelas: \texttt{DRIVER\_HELMET} (pengendara menggunakan helm), \texttt{DRIVER\_NO\_HELMET} (pengendara tanpa helm di atas motor), dan \texttt{MOTORCYCLE} (sepeda motor). Penting untuk dicatat bahwa kelas \texttt{DRIVER\_NO\_HELMET} secara langsung merepresentasikan ``pengendara motor tanpa helm'', sehingga deteksi kelas ini sudah menjadi sinyal pelanggaran tanpa memerlukan asosiasi geometris tambahan antara pengendara dan sepeda motor.

Video uji yang digunakan untuk \textit{benchmarking} pipeline memiliki resolusi 1280$\times$720 piksel pada 25 FPS dengan total durasi 78,8 menit (118.253 frame).


\section{Dataset dan Strategi Augmentasi}
Dataset yang digunakan dalam penelitian ini terdiri dari total 856 citra yang diambil dari rekaman CCTV lingkungan kampus. Distribusi data dibagi menjadi tiga himpunan bagian: \textit{Training Set} (82\%), \textit{Validation Set} (9\%), dan \textit{Test Set} (9\%), sebagaimana dirincikan pada Tabel \ref{tab:dataset_dist}.

\begin{table}[H]
\centering
\caption{Distribusi Dataset Penelitian}
\label{tab:dataset_dist}
\begin{tabular}{lcc}
\hline
\textbf{Himpunan Data} & \textbf{Jumlah Citra} & \textbf{Persentase} \\ \hline
Training Set & 699 & 82\% \\
Validation Set & 79 & 9\% \\
Test Set & 78 & 9\% \\ \hline
\textbf{Total} & \textbf{856} & \textbf{100\%} \\ \hline
\end{tabular}
\end{table}

\subsection{Preprocessing dan Augmentasi Data}
Untuk meningkatkan kemampuan generalisasi model dan mencegah \textit{overfitting}, diterapkan strategi \textit{preprocessing} dan augmentasi data yang ketat.
\begin{enumerate}
    \item \textbf{Preprocessing}:
    \begin{itemize}
        \item \textit{Auto-Orient}: Penyesuaian orientasi citra otomatis.
        \item \textit{Resize}: Pengubahan ukuran citra menjadi 640$\times$640 piksel (\textit{fit within}) untuk menyesuaikan \textit{input layer} YOLOv8.
    \end{itemize}
    \item \textbf{Augmentasi}: Setiap citra pelatihan menghasilkan 3 variasi output (\textit{outputs per training example: 3}) dengan transformasi berikut:
    \begin{itemize}
        \item \textit{Rotation}: Antara -8$^{\circ}$ hingga +8$^{\circ}$.
        \item \textit{Shear}: $\pm$10$^{\circ}$ horizontal dan vertikal.
        \item \textit{Exposure}: Variasi kecerahan antara -10\% hingga +10\%.
        \item \textit{Blur}: \textit{Gaussian blur} hingga 1.4px.
        \item \textit{Noise}: Penambahan \textit{noise} hingga 1.09\% piksel.
        \item \textit{Motion Blur}: Simulasi gerak (panjang 10px, sudut 0$^{\circ}$).
    \end{itemize}
\end{enumerate}

% ==========================================================
\section{Hasil Hyperparameter Tuning Model YOLOv8}

\subsection{Hasil Eksplorasi Hyperparameter}
Proses \textit{hyperparameter tuning} dilakukan untuk mengevaluasi pengaruh variasi parameter pelatihan terhadap kinerja model YOLOv8n (\textit{nano}). Arsitektur YOLOv8n dipilih dengan pertimbangan efisiensi komputasi pada perangkat dengan sumber daya terbatas (2 GB VRAM). Tabel~\ref{tab:tuning} menyajikan ringkasan hasil eksperimen.

\begin{table}[H]
\centering
\caption{Hasil Eksplorasi Hyperparameter YOLOv8}
\label{tab:tuning}
\begin{tabular}{ccccccc}
\hline
\textbf{LR} & \textbf{Batch} & \textbf{Epoch} & \textbf{Precision} & \textbf{Recall} & \textbf{mAP@50} & \textbf{mAP@50:95} \\ \hline
0.001 & 16 & 50  & -- & -- & -- & -- \\
0.001 & 16 & 100 & -- & -- & -- & -- \\
0.01  & 16 & 100 & -- & -- & -- & -- \\
0.01  & 16 & 300 & \textbf{0.875} & \textbf{0.865} & \textbf{0.914} & -- \\ \hline
\end{tabular}
\end{table}

Konfigurasi dengan \textit{learning rate} 0.01, \textit{batch size} 16, dan 300 \textit{epoch} menghasilkan kinerja terbaik dengan mAP@50 sebesar \textbf{91,4\%} pada \textit{validation set}. Grafik pelatihan menunjukkan bahwa seluruh komponen \textit{loss} (box, classification, distribution focal loss) konvergen secara stabil setelah epoch ke-150 tanpa indikasi \textit{overfitting}.

\subsection{Pemilihan Konfigurasi Model}
Berdasarkan hasil eksplorasi, konfigurasi model terpilih dirangkum pada Tabel~\ref{tab:config}.

\begin{table}[H]
\centering
\caption{Konfigurasi Model Terpilih}
\label{tab:config}
\begin{tabular}{ll}
\hline
\textbf{Parameter} & \textbf{Nilai} \\ \hline
Arsitektur & YOLOv8n (\textit{nano}, 6.3 MB) \\
\textit{Learning Rate} & 0.01 \\
\textit{Batch Size} & 16 \\
\textit{Epoch} & 300 \\
\textit{Image Size} & 640 $\times$ 640 \\
\textit{Confidence Threshold} & 0.5 \\
\textit{IoU Threshold (NMS)} & 0.45 \\
Inferensi & CPU (\textit{fallback} dari GPU) \\ \hline
\end{tabular}
\end{table}

% ==========================================================
\section{Evaluasi Kinerja Model Deteksi Helm}

\subsection{Evaluasi Kuantitatif}
Kinerja model deteksi dievaluasi pada \textit{test set} menggunakan metrik \textit{Average Precision} (AP) per kelas. Hasil evaluasi disajikan pada Tabel~\ref{tab:modelperf}.

\begin{table}[H]
\centering
\caption{Kinerja Model Deteksi pada Test Set (mAP@50)}
\label{tab:modelperf}
\begin{tabular}{lc}
\hline
\textbf{Kelas} & \textbf{AP@50} \\ \hline
DRIVER\_HELMET    & 0.870 (87,0\%) \\
DRIVER\_NO\_HELMET & 0.900 (90,0\%) \\
MOTORCYCLE        & 0.940 (94,0\%) \\
\hline
\textbf{Rata-rata (all)} & \textbf{0.900 (90,0\%)} \\ \hline
\end{tabular}
\end{table}

Metrik keseluruhan pada \textit{validation set} menunjukkan: \textbf{Precision = 87,5\%}, \textbf{Recall = 86,5\%}, dan \textbf{mAP@50 = 91,4\%}. Seluruh metrik melampaui target minimal (\textit{precision} $\geq$ 80\%, \textit{recall} $\geq$ 75\%, mAP@50 $\geq$ 80\%).

Temuan penting per kelas:
\begin{enumerate}
    \item \textbf{MOTORCYCLE} (AP = 94,0\%) mencapai akurasi tertinggi karena bentuk visual sepeda motor yang konsisten dan mudah dibedakan.
    \item \textbf{DRIVER\_NO\_HELMET} (AP = 90,0\%) menunjukkan model cukup sensitif dalam mendeteksi pengendara tanpa helm, yang merupakan kelas target utama sistem.
    \item \textbf{DRIVER\_HELMET} (AP = 87,0\%) memperoleh AP terendah, disebabkan variasi warna dan bentuk helm serta potensi oklusi.
\end{enumerate}

\subsection{Distribusi Deteksi pada Video Uji}
Untuk memvalidasi kinerja model pada data video riil, dilakukan pengujian pada 3.000 frame dari rekaman CCTV. Hasil distribusi deteksi disajikan pada Tabel~\ref{tab:detection_dist}.

\begin{table}[H]
\centering
\caption{Distribusi Deteksi pada 3.000 Frame Video Uji}
\label{tab:detection_dist}
\begin{tabular}{lccc}
\hline
\textbf{Kelas} & \textbf{Total Deteksi} & \textbf{Rata-rata/Frame} & \textbf{Proporsi (\%)} \\ \hline
DRIVER\_HELMET    & 4.059 & 1,35 & 43,3 \\
DRIVER\_NO\_HELMET & 1.494 & 0,50 & 15,9 \\
MOTORCYCLE        & 3.822 & 1,27 & 40,8 \\ \hline
\textbf{Total}     & \textbf{9.375} & \textbf{3,12} & 100 \\ \hline
\end{tabular}
\end{table}

Rata-rata 3,12 deteksi per frame menunjukkan bahwa model berhasil mendeteksi objek secara konsisten. Proporsi DRIVER\_NO\_HELMET sebesar 15,9\% dari total deteksi mengindikasikan adanya pelanggaran helm yang terukur pada area pengujian.

\subsection{Analisis Kesalahan Deteksi}
Analisis kesalahan dilakukan secara kualitatif. Penyebab utama kesalahan deteksi:
\begin{enumerate}
    \item \textbf{Oklusi parsial}: Tubuh pengendara terhalang kendaraan lain atau objek statis.
    \item \textbf{Sudut kamera}: Deteksi dari sudut atas atau belakang mengurangi visibilitas area kepala.
    \item \textbf{Kondisi pencahayaan}: Variasi pencahayaan akibat perubahan waktu dan bayangan.
    \item \textbf{Skala objek kecil}: Pengendara pada jarak jauh ($<$32$\times$32 piksel) sulit diklasifikasi.
\end{enumerate}

% ==========================================================
\section{Evaluasi Integrasi YOLOv8 dan ByteTrack}
\label{sec:tracking}

\subsection{Stabilitas Pelacakan Objek}
Evaluasi pelacakan dilakukan pada 3.000 frame video uji (1280$\times$720, 25 FPS). Metrik utama disajikan pada Tabel~\ref{tab:tracking}.

\begin{table}[H]
\centering
\caption{Statistik Pelacakan Objek dengan ByteTrack}
\label{tab:tracking}
\begin{tabular}{lc}
\hline
\textbf{Metrik} & \textbf{Nilai} \\ \hline
Total \textit{Track ID} Unik & 135 \\
Rata-rata Track Aktif per Frame & 3,05 \\
Maksimum Track Aktif Simultan & 11 \\
\textit{Track Activation Threshold} & 0,25 \\
\textit{Lost Track Buffer} & 30 frame ($\sim$2 detik) \\
\textit{Minimum Matching Threshold} & 0,8 \\ \hline
\end{tabular}
\end{table}

ByteTrack berhasil mempertahankan 135 \textit{track ID} unik selama pemrosesan 3.000 frame, dengan rata-rata 3,05 track aktif per frame. Parameter \texttt{lost\_track\_buffer=30} memungkinkan sistem mempertahankan identitas objek selama oklusi singkat ($\sim$2 detik pada 15 FPS).

\subsection{Dampak Threshold Temporal}
\label{sec:threshold}
Pengaruh \textit{threshold} temporal $N$ frame dianalisis menggunakan segmen video yang mengandung pelanggaran (frame 750--999). Mekanisme ini mensyaratkan deteksi \texttt{DRIVER\_NO\_HELMET} harus konsisten selama $N$ frame berturut-turut sebelum dikonfirmasi sebagai pelanggaran. Hasil pengujian disajikan pada Tabel~\ref{tab:threshold}.

\begin{table}[H]
\centering
\caption{Dampak Threshold Temporal terhadap Deteksi Pelanggaran}
\label{tab:threshold}
\begin{tabular}{ccccc}
\hline
\textbf{$N$} & \textbf{Pelanggaran} & \textbf{Confidence Rata-rata} & \textbf{Frame Konfirmasi (Track 4)} & \textbf{Frame Konfirmasi (Track 7)} \\ \hline
1 & 2 & 0,720 & 780 & 834 \\
3 & 2 & 0,764 & 782 & 836 \\
5 & 2 & 0,860 & 789 & 838 \\ \hline
\end{tabular}
\end{table}

Hasil analisis menunjukkan:
\begin{itemize}
    \item \textbf{$N=1$}: Konfirmasi tercepat (frame 780) namun \textit{confidence} rata-rata terendah (0,720) karena deteksi pertama yang sering memiliki confidence rendah langsung diterima.
    \item \textbf{$N=3$} (konfigurasi terpilih): Memberikan keseimbangan antara kecepatan konfirmasi (hanya +2 frame dari $N=1$) dan peningkatan confidence (+6,1\% dibanding $N=1$). Tambahan latency konfirmasi $\sim$244 ms (2 frame $\times$ $\sim$122 ms) tidak signifikan.
    \item \textbf{$N=5$}: Menghasilkan \textit{confidence} tertinggi (0,860) karena hanya mengakumulasi deteksi yang sangat konsisten, namun dengan delay konfirmasi lebih panjang (+9 frame dari $N=1$, $\sim$610 ms tambahan).
\end{itemize}

Seluruh konfigurasi $N$ berhasil mendeteksi kedua pelanggaran pada segmen uji ini (track 4 dan track 7), menunjukkan bahwa mekanisme konfirmasi temporal tidak mengurangi \textit{true positive rate} pada kasus pengendara yang melintasi area kamera dengan kecepatan normal. Nilai $N=3$ dipilih sebagai konfigurasi akhir karena menyeimbangkan \textit{confidence} dan responsivitas.

% ==========================================================
\section{Evaluasi Kinerja Sistem Streaming}

\subsection{Throughput Sistem}
\label{sec:throughput}
Kinerja throughput diukur pada 3.000 frame video uji dengan inferensi CPU. Hasil pengukuran disajikan pada Tabel~\ref{tab:throughput}.

\begin{table}[H]
\centering
\caption{Throughput Sistem Pipeline Deteksi}
\label{tab:throughput}
\begin{tabular}{lcc}
\hline
\textbf{Metrik} & \textbf{Nilai} & \textbf{Keterangan} \\ \hline
FPS Rata-rata & 8,08 & Dari 3.000 frame \\
FPS Median & 8,39 & Nilai tengah \\
FPS Minimum & 1,28 & Spike awal (\textit{cold start}) \\
FPS Maksimum & 9,33 & Performa puncak \\
FPS P5 & 5,64 & Batas bawah 5\% terburuk \\
FPS P95 & 8,91 & Batas atas 95\% \\
Deviasi Standar & -- & Variabilitas FPS \\
Total Waktu & 388,36 detik & Untuk 3.000 frame \\ \hline
\end{tabular}
\end{table}

Throughput rata-rata \textbf{8,08 FPS} pada CPU menunjukkan bahwa sistem mampu memproses video secara \textit{near real-time} meskipun di bawah target 15 FPS (sesuai input CCTV 25 FPS). Nilai P5 sebesar 5,64 FPS menunjukkan performa terburuk masih cukup tinggi untuk deteksi pelanggaran.

Catatan penting:
\begin{enumerate}
    \item \textbf{Penyebab}: Throughput dibatasi oleh CPU Intel i5-5200U (2 core, generasi ke-5) karena GPU tidak kompatibel.
    \item \textbf{Potensi}: GPU kompatibel (CUDA $\geq$ 7.0, misal GTX 1650) diperkirakan meningkatkan throughput 5--10$\times$ lipat (40--80 FPS).
    \item \textbf{Mitigasi}: \textit{Frame skipping} (1 dari 2--3 frame) dapat diterapkan pada produksi tanpa mengurangi akurasi.
\end{enumerate}

\subsection{Latency End-to-End}
\label{sec:latency}
Latency per komponen diukur menggunakan \texttt{time.perf\_counter()} pada setiap tahap pipeline. Rincian latency disajikan pada Tabel~\ref{tab:latency}.

\begin{table}[H]
\centering
\caption{Latency Per Komponen Pipeline (ms) — 3.000 Frame}
\label{tab:latency}
\begin{tabular}{lcccc}
\hline
\textbf{Komponen} & \textbf{Mean} & \textbf{Median} & \textbf{P95} & \textbf{P99} \\ \hline
Inferensi YOLOv8 (CPU)  & 126,48 & 117,59 & 174,90 & 280,58 \\
Pelacakan ByteTrack     & 1,46   & 1,63   & 2,89   & 4,09 \\
Logika Pelanggaran      & 0,03   & 0,02   & 0,10   & -- \\
\hline
\textbf{Total Pipeline} & \textbf{127,98} & \textbf{119,13} & \textbf{177,17} & \textbf{281,51} \\ \hline
\end{tabular}
\end{table}

Analisis latency menunjukkan beberapa temuan penting:

\begin{enumerate}
    \item \textbf{Bottleneck}: Inferensi YOLOv8 menyumbang \textbf{98,8\%} dari total latency pipeline (126,48 ms dari 127,98 ms). Hal ini mengkonfirmasi bahwa optimasi inferensi (GPU, quantization, atau batch processing) akan memberikan dampak terbesar pada performa keseluruhan.
    
    \item \textbf{ByteTrack efisien}: Pelacakan hanya memerlukan \textbf{1,46 ms} rata-rata ($\sim$1,1\% total latency), menunjukkan bahwa algoritma ByteTrack sangat efisien dan tidak menjadi \textit{bottleneck}.
    
    \item \textbf{Logika pelanggaran negligible}: Pengecekan pelanggaran hanya \textbf{0,03 ms} ($\sim$0,02\% total latency), membuktikan bahwa logika berbasis \textit{lookup} dan \textit{counter} sangat ringan secara komputasi.
    
    \item \textbf{Variabilitas}: P99 latency (281,51 ms) signifikan lebih tinggi dari median (119,13 ms), kemungkinan disebabkan oleh \textit{context switching} CPU atau \textit{thermal throttling} pada perangkat laptop.
    
    \item \textbf{Latency real-time}: Latency median \textbf{119,13 ms} ($<$1 detik) memenuhi kebutuhan sistem \textit{near real-time} untuk deteksi pelanggaran.
\end{enumerate}

% ==========================================================
% ==========================================================
\section{Evaluasi Kinerja Sistem Streaming (Kafka \& Spark)}
\label{sec:streaming_eval}

Evaluasi kinerja sistem \textit{backend} streaming dilakukan untuk memastikan skalabilitas dan stabilitas pipeline data dari detektor hingga dashboard. Komponen yang dievaluasi meliputi \textit{throughput} pengiriman pesan oleh Kafka Producer dan latensi pemrosesan mikro-batch oleh Apache Spark.

\subsection{Benchmark Kafka Producer}
Pengujian \textit{load testing} dilakukan dengan mengirimkan 5.000 pesan simulasi pelanggaran secara beruntun ke topik \texttt{video.violations}. Hasil pengujian disajikan pada Tabel~\ref{tab:kafka_bench}.

\begin{table}[H]
\centering
\caption{Kinerja Kafka Producer (5.000 Pesan)}
\label{tab:kafka_bench}
\begin{tabular}{lc}
\hline
\textbf{Metrik} & \textbf{Nilai} \\ \hline
Total Pesan Terkirim & 5.000 \\
Total Waktu Eksekusi & 1,34 detik \\
\textbf{Throughput Rata-rata} & \textbf{3.728 pesan/detik} \\
Latency Rata-rata per Pesan & 0,241 ms \\
Latency P99 & 2,703 ms \\ \hline
\end{tabular}
\end{table}

Dengan throughput mencapai \textbf{3.728 pesan/detik}, sistem Kafka sangat memadai untuk menangani beban dari satu kamera (maksimal 15--25 pesan/detik jika semua objek melanggar) maupun skenario multi-kamera (misal: 100 kamera $\times$ 25 FPS = 2.500 pesan/detik). Latency pengiriman 0,24 ms dapat diabaikan (\textit{negligible}) dibandingkan latency inferensi model (~126 ms).

\subsection{Kinerja Pemrosesan Spark Structured Streaming}
Apache Spark dikonfigurasi dengan \textit{trigger interval} 5 detik. Observasi log pemrosesan menunjukkan karakteristik sebagai berikut:

\begin{itemize}
    \item \textbf{Latency Batch}: Rata-rata waktu pemrosesan satu mikro-batch adalah 3--5 detik.
    \item \textbf{Latency End-to-End}: Total waktu dari deteksi hingga muncul di dashboard berkisar antara 5--10 detik.
    \item \textbf{Dedupikasi}: Waktu \textit{watermark} 30 detik efektif mencegah duplikasi data akibat \textit{network retry}, memastikan setiap pelanggaran unik hanya dicatat satu kali per \textit{track\_id}.
\end{itemize}

% ==========================================================
\section{Evaluasi Data Mart dan Dashboard Visualisasi}
\label{sec:dashboard}

\subsection{Implementasi Star Schema}
Data Mart dibangun di atas PostgreSQL menggunakan arsitektur \textit{Star Schema} untuk mendukung performa kueri analitik. Skema terdiri atas:
\begin{enumerate}
    \item \textbf{Tabel Fakta (\texttt{fact\_violations})}: Menyimpan detail kejadian (track\_id, confidence, bounding box).
    \item \textbf{Dimensi Waktu}: Di-\textit{denormalisasi} langsung ke tabel fakta (kolom \texttt{hour}, \texttt{day\_of\_week}, \texttt{time\_period}) untuk mempercepat agregasi grafik tren tanpa \textit{join} yang mahal.
    \item \textbf{Dimensi Kamera}: Tabel \texttt{dim\_camera} untuk metadata lokasi gerbang dan petugas jaga.
\end{enumerate}

Pengujian integrasi membuktikan bahwa data pelanggaran yang dikirim dari pipeline deteksi berhasil disimpan dengan integritas penuh, termasuk atribut waktu lokal (WIB) dan periode waktu (Pagi/Siang/Sore/Malam).

\subsection{Visualisasi Grafana}
Dashboard pemantauan keselamatan (\textit{Safety Dashboard}) menyajikan enam panel utama untuk mendukung pengambilan keputusan:

\begin{figure}[H]
\centering
% \includegraphics[width=\textwidth]{grafana_dashboard} % Placeholder
\caption{Tampilan Antarmuka ITERA Smart Sentinel Dashboard}
\label{fig:dashboard}
\end{figure}

\begin{enumerate}
    \item \textbf{KPI Overview}: Menampilkan total pelanggaran hari ini, rata-rata \textit{confidence}, dan rata-rata \textit{processing latency} secara \textit{real-time}.
    \item \textbf{Tren Temporal}: Grafik deret waktu (\textit{time series}) yang menunjukkan lonjakan pelanggaran per jam selama 24 jam terakhir.
    \item \textbf{Heatmap Aktivitas}: Matriks Jam $\times$ Hari yang memvisualisasikan "jam rawan" pelanggaran (misal: Senin pagi jam 07.00--08.00).
    \item \textbf{Distribusi Lokasi}: Diagram batang yang membandingkan jumlah pelanggaran antar gerbang/kamera.
    \item \textbf{Distribusi Waktu}: Diagram lingkaran (\textit{pie chart}) proporsi pelanggaran berdasarkan periode waktu (Pagi/Siang/Sore/Malam).
    \item \textbf{Log Detail}: Tabel daftar pelanggaran terbaru lengkap dengan foto bukti (melalui referensi ID) dan tingkat kepercayaan model.
\end{enumerate}

Evaluasi subjektif menunjukkan bahwa dashboard mampu menyajikan insight dalam waktu $<10$ detik setelah kejadian, memenuhi kriteria sistem \textit{near real-time monitoring}.

% ==========================================================
\section{Pembahasan Hasil Akhir}

\subsection{Tujuan 1: Model Deteksi Helm}
Model YOLOv8n terpilih mencapai mAP@50 \textbf{91,4\%} (val) dan \textbf{90,0\%} (test). Kelas target utama \texttt{DRIVER\_NO\_HELMET} terdeteksi dengan presisi tinggi (AP 90\%). Penerapan logika \textit{temporal consistency} ($N=3$) berhasil mengeliminasi \textit{false positive} akibat oklusi sesaat, meningkatkan rata-rata confidence dari 0,72 menjadi 0,76.

\subsection{Tujuan 2: Kinerja Pipeline System}
Secara keseluruhan, sistem mencapai throughput \textbf{3.728 pesan/detik} (Kafka) dan frame rate pemrosesan visual \textbf{8,08 FPS} (CPU). Meskipun inferensi visual menjadi \textit{bottleneck} utama (126 ms/frame) akibat keterbatasan GPU, arsitektur \textit{decoupled} menggunakan Kafka memungkinkan penskalaan horizontal di masa depan. Jika menggunakan GPU modern (T4/RTX), throughput visual diproyeksikan dapat menyamai throughput Kafka yang sangat tinggi.

\subsection{Tujuan 3: Dashboard Keputusan}
Integrasi Spark Streaming dan Grafana berhasil menyajikan data terstruktur dalam format visual yang mudah dipahami. Fitur \textit{heatmap} dan \textit{trend analysis} secara langsung menjawab kebutuhan UPT K3L untuk mengidentifikasi pola pelanggaran, sehingga patroli keamanan dapat dilakukan secara terarah pada jam dan lokasi rawan.

\subsection{Keterbatasan Penelitian}
\begin{enumerate}
    \item \textbf{Inferensi CPU}: Ketergantungan pada CPU membatasi FPS. Migrasi ke \textit{edge device} dengan NPU (misal Jetson Nano) disarankan.
    \item \textbf{Variasi Cuaca}: Pengujian belum mencakup kondisi hujan lebat atau kabut tebal.
    \item \textbf{Retensi Data}: Kebijakan penghapusan data lama (\textit{data retention}) belum diimplementasikan otomatis pada database.
\end{enumerate}

% ==========================================================
\section{Ringkasan Temuan Utama}

\begin{table}[H]
\centering
\caption{Ringkasan Temuan Penelitian Akhir}
\label{tab:summary_final}
\begin{tabular}{lp{9cm}}
\hline
\textbf{Aspek} & \textbf{Temuan Utama} \\ \hline
Akurasi Deteksi & mAP@50 = 91,4\%. AP \texttt{DRIVER\_NO\_HELMET} = 90\%. \\
Responsivitas & Latency Pipeline Visual: 128 ms (CPU). Latency Streaming: 5--10 detik (End-to-End). \\
Skalabilitas & Kafka Producer mampu menangani 3.728 pesan/detik, sangat memadai untuk ekspansi multi-kamera. \\
Stabilitas & Pipeline berjalan stabil pada throughput 8 FPS tanpa \textit{memory leak} yang teramati. \\
Nilai Bisnis & Dashboard menyajikan heatmap jam rawan dan tren harian secara otomatis tanpa rekapitulasi manual. \\ \hline
\end{tabular}
\end{table}
